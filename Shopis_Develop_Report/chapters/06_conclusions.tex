\chapter{Déploiement and CI/CD}
\label{ch:con}
\section{Déploiement web (Vercel, Netlify, Astro)}
Typically a conclusions chapter first summarizes the investigated problem and its aims and objectives. It summaries the critical/significant/major findings/results about the aims and objectives that have been obtained by applying the key methods/implementations/experiment set-ups. A conclusions chapter draws a picture/outline of your project's central and the most signification contributions and achievements. 

A good conclusions summary could be approximately 300--500 words long, but this is just a recommendation.

A conclusions chapter followed by an abstract is the last things you write in your project report.

\section{Déploiement mobile (Expo, Play Store, App Store)}
This section should refer to Chapter~\ref{ch:results} where the author has reflected their criticality about their own solution. Concepts for future work are then sensibly proposed in this section.

\textbf{Guidance on writing future work:} While working on a project, you gain experience and learn the potential of your project and its future works. Discuss the future work of the project in technical terms. This has to be based on what has not been yet achieved in comparison to what you had initially planned and what you have learned from the project. Describe to a reader what future work(s) can be started from the things you have completed. This includes identifying what has not been achieved and what could be achieved. 



A good future work summary could be approximately 300--500 words long, but this is just a recommendation.


\section{Automatisation (GitHub Actions / Bolt pipelines)}
This section should refer to Chapter~\ref{ch:results} where the author has reflected their criticality about their own solution. Concepts for future work are then sensibly proposed in this section.

\textbf{Guidance on writing future work:} While working on a project, you gain experience and learn the potential of your project and its future works. Discuss the future work of the project in technical terms. This has to be based on what has not been yet achieved in comparison to what you had initially planned and what you have learned from the project. Describe to a reader what future work(s) can be started from the things you have completed. This includes identifying what has not been achieved and what could be achieved. 



A good future work summary could be approximately 300--500 words long, but this is just a recommendation.


\section{Gestion des environnements (dev / prod / test)}
This section should refer to Chapter~\ref{ch:results} where the author has reflected their criticality about their own solution. Concepts for future work are then sensibly proposed in this section.

\textbf{Guidance on writing future work:} While working on a project, you gain experience and learn the potential of your project and its future works. Discuss the future work of the project in technical terms. This has to be based on what has not been yet achieved in comparison to what you had initially planned and what you have learned from the project. Describe to a reader what future work(s) can be started from the things you have completed. This includes identifying what has not been achieved and what could be achieved. 



A good future work summary could be approximately 300--500 words long, but this is just a recommendation.